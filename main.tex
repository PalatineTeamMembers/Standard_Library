\documentclass[titlepage,landscape,a4paper,10pt]{article}
\usepackage{listings, color, fontspec, minted, setspace, titlesec, fancyhdr, dingbat, mdframed, multicol}
\usepackage{graphicx, amssymb, amsmath, textcomp, booktabs}
\usepackage[Chinese]{ucharclasses}
\usepackage[left=1.5cm, right=0.7cm, top=1.7cm, bottom=0.0cm]{geometry}

%configure the top corners
\pagestyle{fancy}
\setlength{\headsep}{0.1cm}
\rhead{Page \thepage}
\lhead{上海交通大学 Shanghai Jiao Tong University}

%configure space between the two columns
\setlength{\columnsep}{30pt}

%configure fonts
\setmonofont{Isotype}[Scale=0.8]
\newfontfamily\substitutefont{SimHei}[Scale=0.8]
\setTransitionsForChinese{\begingroup\substitutefont}{\endgroup}

%configure minted to display codes 
\definecolor{Gray}{rgb}{0.9,0.9,0.9}

%remove leading numbers in table of contents
\setcounter{secnumdepth}{0}

%configure section style
%\titleformat{\section}
%	{\normalfont\normalsize}	% The style of the section title
%	{}					% a prefix
%	{0pt}				% How much space exists between the prefix and the title
%	{\quad}				% How the section is represented
\titleformat{\section}{\large}{}{0pt}{}
\titlespacing{\section}{0pt}{0pt}{0pt}

%enable section to start new page automatically
%\let\stdsection\section
%\renewcommand\section{\penalty-100\vfilneg\stdsection}

%\renewcommand\theFancyVerbLine{\arabic{FancyVerbLine}}
\renewcommand{\theFancyVerbLine}{\sffamily \textcolor[rgb]{0.5,0.5,0.5}{\scriptsize {\arabic{FancyVerbLine}}}}

\setminted[cpp]{
	style=xcode,
	mathescape,
	linenos,
	autogobble,
	baselinestretch=0.9,
	tabsize=2,
	fontsize=\normalsize,
	%bgcolor=Gray,
	frame=single,
	framesep=1mm,
	framerule=0.3pt,
	numbersep=1mm,
	breaklines=true,
	breaksymbolsepleft=2pt,
	%breaksymbolleft=\raisebox{0.8ex}{ \small\reflectbox{\carriagereturn}}, %not moe!
	%breaksymbolright=\small\carriagereturn,
	breakbytoken=false,
}
\setminted[java]{
	style=xcode,
	mathescape,
	linenos,
	autogobble,
	baselinestretch=1.0,
	tabsize=2,
	%bgcolor=Gray,
	frame=single,
	framesep=1mm,
	framerule=0.3pt,
	numbersep=1mm,
	breaklines=true,
	breaksymbolsepleft=2pt,
	%breaksymbolleft=\raisebox{0.8ex}{ \small\reflectbox{\carriagereturn}}, %not moe!
	%breaksymbolright=\small\carriagereturn,
	breakbytoken=false,
}
\setminted[text]{
	style=xcode,
	mathescape,
	linenos,
	autogobble,
	baselinestretch=1.0,
	tabsize=2,
	%bgcolor=Gray,
	frame=single,
	framesep=1mm,
	framerule=0.3pt,
	numbersep=1mm,
	breaklines=true,
	breaksymbolsepleft=2pt,
	%breaksymbolleft=\raisebox{0.8ex}{ \small\reflectbox{\carriagereturn}}, %not moe!
	%breaksymbolright=\small\carriagereturn,
	breakbytoken=false,
}

%configure titles

%THE SCL BEGINS
\begin{document}

\begin{multicols*}{2}

\begin{spacing}{0}
	\tableofcontents
\end{spacing}
\end{multicols*}

\begin{multicols}{2}

\newpage
\begin{spacing}{0.8}

\section{数据结构}

\subsection{带插入k小值:替罪羊树套线段树}
\inputminted{cpp}{datastructure/kthElementWithInsertion.cpp}

\subsection{动态树}
\inputminted{cpp}{datastructure/linkCutTree_lty.cpp}

\subsection{可持久化Treap}
\inputminted{cpp}{datastructure/persistentTreap.cpp}

\subsection{点分治}
\inputminted{cpp}{datastructure/treeDivideConquer_lty.cpp}

\subsection{左偏树}
\inputminted{cpp}{datastructure/leftSkewedTree.cpp}

\section{数学}
\subsection{高斯消元}
\inputminted{cpp}{math/gause.cpp}

\section{图论}
\subsection{欧拉路径}
\inputminted{cpp}{graphTheory/Fleury.cpp}

\subsection{平面图转对偶图}
\inputminted{cpp}{graphTheory/planar2dual.cpp}

\subsection{Tarjan}
\inputminted{cpp}{graphTheory/Tarjan.cpp}

\subsection{2-Set}
\inputminted{cpp}{graphTheory/twoSet.cpp}

\section{字符串}

\subsection{后缀自动机}
\inputminted{cpp}{string/sam.cpp}

\subsection{回文自动机}
\inputminted{cpp}{string/pam.cpp}

\section{动态规划}
\subsection{插头DP}
\inputminted{cpp}{DP/plugDp_lty.cpp}
\end{spacing}

\end{multicols}

\end{document}

